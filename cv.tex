\documentclass[a4paper,12pt]{article}
\usepackage[utf8]{inputenc}
\usepackage[english,russian]{babel}
\usepackage{amsmath}
\usepackage{a4wide}
\usepackage{indentfirst}
\usepackage{color}
\usepackage{enumitem}
\usepackage{graphicx}
\usepackage{hyperref}

\definecolor{darkblue}{rgb}{0,0,0.5}

\hypersetup{
  unicode=true,
  colorlinks=true,
  urlcolor=darkblue
}

\pagestyle{empty}

\newcommand{\SectionTemplate}[2]{\bigskip\par\noindent{\large\textbf{#1}}\hfill#2\par}
\newcommand{\SectionQ}[4]{\SectionTemplate{#1}{\raisebox{#4}[0pt][0pt]{\includegraphics[scale=#3]{pics/#2}}}}
\newcommand{\Section}[1]{\SectionTemplate{#1}{}}
\newcommand{\SectionComment}[1]{\noindent{\footnotesize\textbf{#1}}}
\newcommand{\alternative}[1]{\textcolor[rgb]{0.5,0.5,0.5}{#1}}
\newcommand{\mailto}[1]{\href{mailto:#1}{#1}}
\newenvironment{items}{\begin{itemize}[label=]}{\end{itemize}}
\newenvironment{subitems}{\begin{itemize}[topsep=0pt]}{\end{itemize}}


\begin{document}

% Контактные данные, страна/город проживания
\SectionQ{Personal Information}{me.jpg}{0.25}{-3.5cm}
\begin{items}
  \item Name: Andrei Matveiakin \alternative{(Андрей Матвеякин).}

  \begin{tabular}{@{}ll}
    Address: & Schaufelbergerstrasse 57, \\
             & 8055 Zürich, Switzerland \\
    Phone: & +41~79~315-04-02 \\
    E-mail \& Hangouts: & \mailto{matveyakin@gmail.com} \\
    Skype: & matveyakin
  \end{tabular}
\end{items}


% Университет, даты обучения, специальность, уровень (master). Также можете включить туда любые дополнительные курсы
\Section{Education}
\begin{items}
  \item In 2012 I graduated with Honors
  from \href{http://www.msu.ru/en/}{Lomonosov Moscow State University}
  with the qualification of Specialist in Mathematics
  and specialization in Computational Mathematics.
\end{items}


% Название компании, даты работы там, Ваша должность, краткое описание деятельности
\Section{Work Experience}
\begin{items}
  \item For the last 3.5 years
  I've been working as software developer in \href{http://rfdyn.com/}{Rock Flow Dynamics},
  writing a petroleum reservoir simulator called tNavigator.

%   TODO: wording
%   \item These are the major tasks I have been performing (in groups of 3 to 6 people):
  \item These are the major problems I have been solving (in groups of 3 to 6 people):
  \begin{subitems}
    \item Module that allows to watch and manage calculations on a remote cluster;
    \item Support for variables and conditional expressions in input language;
    \item Alkaline, surfactant and polymer injection simulation;
    \item And the main thing---history matching module. In a nutshell history matching is an inverse problem to the equation that tNavigator usually solves. The aim is to find a set of coefficients for which the solution minimizes a functional describing deviation between calculated and historical values. To solve this problem we had to deal with a bunch of tasks, such as developing a computational queue manager, implementing and tuning minimization algorithms and writing GUI that displays comparative graphs and diagrams on basis of computational results, which are likely not fitting in RAM.
  \end{subitems}
\end{items}


% Ваши основные языки программирования, операционные системы и тд
\Section{Technologies}
\begin{items}
  \item I mainly program in C++ trying to keep an eye on new language features; \\
  write scripts in Bash and Python when necessary. \\
  Have also studied D and Haskell, but haven't used them in real life yet.

  \item I am familiar with C++ Standard Library and Qt and \\
  have some experience with Boost and OpenGL.
\end{items}
% TODO: OS, etc.
% TODO: OpenCV, SFML - ?
\newpage

% NOTE:  I've splitted it into 2 parts % TODO: What it ok?
% Additional information / Awards and achievements
% - Информация о вашей работе над open-source проектами (можно с ссылками)
% - Опыт и результаты участия в олимпиадах по математике и информатике (даже со школьных времен)
% - Любые проекты, сделанные "для себя" самостоятельно или с друзьями
\Section{Projects Besides Work}   % TODO: Rename header
\begin{items}
  \item I'm used to writing utils for my daily tasks when there is not suitable solution, e.\,g.
  \begin{subitems}
    \item I wrote a \href{https://github.com/amatveyakin/area-measurement}{tool} that allows to quickly measure lengths and areas of objects on a 2D picture and export printable result;
    \item Another example is \href{https://github.com/amatveyakin/random-chess-helper}{mobile app} (initially written for Symbian 60, then ported to Android) that generates positions for Chess-960 (aka Fischer Random Chess).
  \end{subitems}

  \item Several years ago my friend Alexey Eremin and I wrote a couple of video games:
  \begin{subitems}
    \item A split screen \href{https://github.com/amatveyakin/crazy-tetris}{tetris-like game} where several players battle each other using bonuses that occasionally appear in playing field;
    \item A demo of \href{https://github.com/amatveyakin/cubricate}{minecraft-like game} with raytracing.
  \end{subitems}

  \item Now I prefer to do something more useful, i.\,e. contribute open source projects, mostly
  \begin{subitems}
    \item \href{http://www.krusader.org/}{Krusader}: improve queue manager, add drag\&drop target highlighting, add new actions, fix several bugs and leaks, do some speed-up, clean up code;
    \item \href{http://kate-editor.org/}{Kate}: add block editing, enhance search interface.
  \end{subitems}
\end{items}


\Section{Awards and Achievements}
\begin{items}
  \item
  My most noticeable achievements in competitive programming and mathematics are:
  \begin{subitems}
    \item All-Russian Olympiad in Informatics III prize in 11 grade,
    \item All-Russian Olympiad in Informatics III prize in 10 grade,
    \item Krasnodar Territory Olympiad in Mathematics I prize in 11 grade, and
    \item Multiple awards in the \href{http://en.wikipedia.org/wiki/Tournament_of_the_Towns}{Tournament of the Towns}, best result---22.5 points.
  \end{subitems}
\end{items}


% Языки, которыми Вы владеете
\Section{Languages}
\begin{items}
  \item
  \begin{subitems}
    \item Russian (mother tongue),
    \item English (fluently),
    \item German (basic, B1).
  \end{subitems}
\end{items}


\vfill
\begin{raggedleft}
  \scriptsize\textsl{Compilation date — May 10, 2016}   % TODO: Set date automatically

\end{raggedleft}

\end{document}
